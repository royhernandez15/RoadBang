%%%%%%%%%%%%%%%%%%%%%%%%%%%%%%%%%%%%%%%%%%%%%%%%
\input{./format/preamble.ltx} 

%%%%%%%%%%%%%%%%%%%%%%%%%%%%%%%%%%%%%%%%%%%%%%%%
% as needed, comment the following lines by prefixing the percent sign (%) at the start of the line

\Drafttrue % comment to disable putting some guides in the draft form of the document

% \PutLineNumberstrue % comment to disable line numbers and certain  preparation guides 

\Figurestrue % comment to disable the rendering figures

\GroupIDtrue % comment to disable group ID

\ResultDiscusstrue % comment to disable results and discussions

\Conctrue % comment to disable conclusions

\Finishedtrue % comment to disable manuscript for final defense or binding/submission

\ApprovalSheetSignedtrue % comment to disable inclusion of the signed approval sheet

% \Gradtrue % comment to disable graduate school format

% \PhDtrue % comment to disable PhD dissertation format

% \PubListtrue % comment to disable publication list

\Vitatrue % comment to disable author(s) vita

\Indextrue % comment to disable index 

%%%%%%%%%%%%%%%%%%%%%%%%%%%%%%%%%%%%%%%%%%%%%%%%
% document IDs

% specify if dissertation, thesis, project, dissertation proposal, thesis proposal, project proposal 
\newcommand{\documentType}{Thesis} 
%\newcommand{\documentType}{Thesis Proposal}
%\newcommand{\documentType}{Dissertation Proposal}
%\newcommand{\documentType}{Dissertation}
\newcommand{\college}{Gokongwei College of Engineering}
\newcommand{\department}{Department of Electronics and Communications Engineering} 
\newcommand{\degreeType}{Bachelor of Science}
%\newcommand{\degreeType}{Bachelor and Master of Science}  
%\newcommand{\degreeType}{Master of Engineering Program} 
%\newcommand{\degreeType}{Master of Science} 
%\newcommand{\degreeType}{Doctor of Philosophy} 
\newcommand{\degree}{Electronics and Communications Engineering}
\newcommand{\degreeAbbrv}{BS-ECE}
%\newcommand{\degreeAbbrv}{BS-MS-ECE}
%\newcommand{\degreeAbbrv}{MEP-ECE}
%\newcommand{\degreeAbbrv}{MS-ECE}
%\newcommand{\degreeAbbrv}{PhD-ECE}

\newcommand{\documentAdviserTitle}{Dr.} 
\newcommand{\documentAdviser}{Francisco D. Baltasar}

\newcommand{\examinerChairTitle}{Dr.} 
\newcommand{\examinerChair}{Amado Z. Hernandez}

% Sort in alphabetically ascending manner the surnames of the examiners
\newcommand{\examinerATitle}{Dr.} 
\newcommand{\examinerA}{Jose Y. Alonzo}

\newcommand{\examinerBTitle}{Dr.} 
\newcommand{\examinerB}{Mariana X. Mercado}

% Note that \examinerC and \examinerD only applies for PhD dissertations
\newcommand{\examinerCTitle}{Dr.} 
\newcommand{\examinerC}{Rafael W. Sison}

\newcommand{\examinerDTitle}{Dr.} 
\newcommand{\examinerD}{Apolinario V. Valenzuela}

% College signatories for graduate theses/dissertations
\newcommand{\RASDTitle}{Dr.} 
\newcommand{\RASDName}{Isabella S. Garcia}

\newcommand{\deanTitle}{Dr.} 
\newcommand{\deanName}{Diego U. Lopez}

\newcommand{\groupID}{ESG-04} % group ID is for undergraduates as of this formatting

\newcommand{\numberOfAuthors}{5} % adapt the number of names below accordingly and sort the sunames in alphabetically ascending manner, like in the following example

\defineAuthor{surname1}{dela Cruz}
\defineAuthor{firstname1}{Juan Z.}

\defineAuthor{surname2}{Franco}
\defineAuthor{firstname2}{Nat Y.}

\defineAuthor{surname3}{Garcia}
\defineAuthor{firstname3}{Sebastian X.}

\defineAuthor{surname4}{Martinez}
\defineAuthor{firstname4}{Isabella W.}

\defineAuthor{surname5}{Rianzares}
\defineAuthor{firstname5}{Max V.}

\newcommand{\documentTitle}{Electrical, Electromagnetic, and Optical Characterization of the InP/InGaAs~Alloy~System} % put tilde (~) between words to indicate non-breaking adjacent words

\newcommand{\keywords}{alloy system, characterization, InP, InGaAs}

\newcommand{\finalDefenseDate}{\usdate\today} % replace ''\usdate\today'' by your final defense date; you may need to use non-breaking space with the use of tildes (~); if so, do not remove the tildes in order to not break the date

\ifGrad
	\newcommand{\scannedApprovalSheetFileName}{./figure/Signed_Thesis_Approval_Sheet_Graduate.pdf} % filename of the signed approval sheet (for graduate)
\else	
	\newcommand{\scannedApprovalSheetFileName}{./figure/Signed_Thesis_Approval_Sheet_Undergraduate.pdf} % filename of the signed approval sheet (for undergraduate)
\fi

\hyphenation{op-tical net-works semi-conduc-tor evi-dent re-la-tive re-si-den-tial po-la-ri-za-tion so-lu-tion/s} % for correcting bad hyphenation

%%%%%%%%%%%%%%%%%%%%%%%%%%%%%%%%%%%%%%%%%%%%%%%%
\input{./format/postamble.ltx} 

%%%%%%%%%%%%%%%%%%%%%%%%%%%%%%%%%%%%%%%%%%%%%%%%
% for placing user-defined-ambles

\DeclareMathAlphabet{\mathitbf}{OML}{cmm}{b}{it} % for math italic bold, but can also use \mathbfit
\newcommand{\redtx}[1]{\textcolor[rgb]{0.65,0.16,0}{#1}} % for formatting text to have a red color
\newcommand{\graytx}[1]{\textcolor[rgb]{0.75,0.75,0.75}{#1}} % for formatting text to have a gray color

%%%%%%%%%%%%%%%%%%%%%%%%%%%%%%%%%%%%%%%%%%%%%%%%
% \includeonly{} is for specifying which files to include; if you only want to work on one or few chapters, you can only include those chapters, which will speed up the document build; advantage: fast if you have a large number of images in your results chapter, which you do not need when you are working on other chapters; you can still reference all the figures in the omitted chapter, as long as you have previously LaTeX-built the entire document

% Note that the file names below must correspond to those names inside \include{} in the \begin{document} ... \end{doument} enviroment, otherwise the chapter will not be included

%  the excludeonly package provides the logically opposite command: \excludeonly{<file list>}

\includeonly{% just comment those portions that you do not want to be included in the parsing
introduction,
literature_review,
theoretical_considerations,
design_considerations,
methodology,
results_and_discussions,
conclusions,
answers_to_questions,
revisions_to_the_proposal,
revisions_to_the_final,
usage_examples,
publication,
vita,
}

%%%%%%%%%%%%%%%%%%%%%%%%%%%%%%%%%%%%%%%%%%%%%%%%
\begin{document}
\pagenumbering{roman} % roman page numbering starts here

%%%%%%%%%%%%%%%%%%%%%%%%%%%%%%%%%%%%%%%%%%%%%%%%
\input{./format/pre_toc.ltx}
\cleardoublepage

%%%%%%%%%%%%%%%%%%%%%%%%%%%%%%%%%%%%%%%%%%%%%%%%
\begin{SingleSpace}
\tableofcontents
\cleardoublepage

%%%%%%%%%%%%%%%%%%%%%%%%%%%%%%%%%%%%%%%%%%%%%%%%
\listoffigures
\cleardoublepage

%%%%%%%%%%%%%%%%%%%%%%%%%%%%%%%%%%%%%%%%%%%%%%%%
\listoftables
\cleardoublepage

%%%%%%%%%%%%%%%%%%%%%%%%%%%%%%%%%%%%%%%%%%%%%%%
\phantomsection
\addcontentsline{toc}{chapter}{Abbreviations}
{
	\printterms[database=abbreviation, style=indexalign, prelocation=dotfill, location=first, columns=1, postname=\hspace{3em}]
	\thispagestyle{plain}
}	
\cleardoublepage

%%%%%%%%%%%%%%%%%%%%%%%%%%%%%%%%%%%%%%%%%%%%%%%
\phantomsection
\addcontentsline{toc}{chapter}{Notation}
{
	\printterms[database=notation, style=indexalign, prelocation=dotfill, location=first, columns=1, postname=\hspace{3em}]
	{
	\vspace{3ex}
	\noindent Throughout this \MakeTextLowercase{\documentType}, mathematical notations conform to ISO~80000-2 standard, e.g. variable names are printed in italics, the only exception being acronyms like e.g. $\mathrm{SNR}$, which are printed in regular font.  Constants are also set in regular font like $\mathrm{j}$.  Functions are also set in regular font, e.g. in $\sin \left( \cdot \right)$.  Commonly used notations are $t$, $f$, $\mathrm{j} = \sqrt{-1}$, $n$ and $\exp \left( \cdot \right)$, which refer to the time variable, frequency variable, imaginary unit, $n$th variable, and exponential function, respectively.
	}
	\thispagestyle{plain}
}
\cleardoublepage

%%%%%%%%%%%%%%%%%%%%%%%%%%%%%%%%%%%%%%%%%%%%%%%%
\phantomsection
\addcontentsline{toc}{chapter}{Glossary}
{
	\printterms[database=glossary, style=indexalign, prelocation=none, location=hide, columns=1, postname=\hspace{3em}]
	\thispagestyle{plain}
}
\cleardoublepage

%%%%%%%%%%%%%%%%%%%%%%%%%%%%%%%%%%%%%%%%%%%%%%%%
\lstlistoflistings
\cleardoublepage
\end{SingleSpace}

%%%%%%%%%%%%%%%%%%%%%%%%%%%%%%%%%%%%%%%%%%%%%%%%
\pagenumbering{arabic} % arabic page numbering starts here
\chapter{Introduction}
\label{ch:intro}
\startcontents[chapters]
\begin{SingleSpace}	
	\Mprintcontents  % for creating an actual mini TOC for this chapter
\end{SingleSpace}
\section{Background of the Study}

Aside from the usual text descriptions of the background, put here figures that will cast images to your audience about the context of your work.

\graytx{\Blindtext}


\section{Prior Studies}

Put here a narrative and a \index{summary}summary (not a duplicate) of your literature review chapter.  In this section, summarize and highlight the gap(s) found in the literature review in Chapter~\ref{ch:litrev}. Preferably, a table showing the summary would be helpful. 

Prior Studies or Literature Review\footnote{The main difference between the Prior Studies and Literature Review is that the Prior Studies is done in a concise manner.  By the way, this is also an example of a footnote usage.} (expansion of the Prior Studies) is basically about \redtx{competition}. \hl{Competition}.

So the \underline{suggested} goals in writing the narrative of the Prior Studies in summative and highlighted forms  are, in no particular order:

\begin{enumerate}
	\item to mention briefly the problem; 

	\item to show the features of the existing literature in solving the problem

	\item to show the weaknesses of the solutions of existing literature 

	\item to show how your solution is better (can be better (for proposals))
\end{enumerate}

\noindent If the suggested table will be placed, please discuss it in light of the above-mentioned items. 

 \graytx{\blindtext}


\section{Problem Statement}

The problem statement needs to be very clear and to the point. 

\noindent A persuasive problem statement from a contextualized and intended-audience-awareness perspective consists of:

\begin{enumerate}
	\item PS1: description of the ideal scenario for your intended audience	
	\begin{itemize}
		\item Describe the goals, desired state, or the values that your audience considers important and that are relevant to the problem.
	\end{itemize}
	
	\item PS2:  reality of the situation
	\begin{itemize}
			\item Describe a condition that prevents the goal, state, or value discussed in PS1 from being achieved or realized at the present time.
			\item It is imperative to make the audience feel the pain point.
	\end{itemize}
	
	\item PS3:  consequences for the audience		
	\begin{itemize}
			\item Using specific details, show how the situation contains little promise of improvement unless something is done.
	\end{itemize}

\end{enumerate}

\noindent After the above-mentioned items, succinctly describe your solution.  Please avoid describing your entire solution here since you will articulate and elucidate it by showing what you want to achieve through your objectives, and how you will make it through your methodology.

\noindent A well constructed problem statement will convince your audience that the problem is real and worth having you solve it.



\graytx{\blindtext}



\section{Objectives}

Your objectives are the states that you desire to achieve in solving the problem. The general objective is the main state to be achieved whereas the specific ones are sub-states to be achieved.

\subsection{General Objective(s)}
To \ldots;

\subsection{Specific Objectives}

\begin{enumerate}
	\item To  \ldots;
	
	\item To  \ldots;
	
	\item To  \ldots;
	
	\item To  \ldots;
	
	\item To  \ldots;
\end{enumerate}



\section{Significance of the Study}

\graytx{\blindtext}



\section{Assumptions, Scope and Delimitations}

Bulletize your assumptions in one group, and then bulletize the scope in another, and do the same for your delimitations. The assumptions to put here are those major facts or statements that are \textit{key} for your proposed solution to work. Scope refers to the space(s) for the operation of your proposed solution, whereas delimitations are the limits of the operation of your proposed solution.

\subsection{Assumptions}

\begin{enumerate}
	\item \ldots;
	
	\item \ldots;
	
	\item \ldots;	
\end{enumerate}

\subsection{Scope}
\begin{enumerate}
	\item \ldots;
	
	\item \ldots;
	
	\item \ldots;	
\end{enumerate}

\subsection{Delimitations}
\begin{enumerate}
	\item \ldots;
	
	\item \ldots;
	
	\item \ldots;	
\end{enumerate}

\section{Description and Methodology of the \documentType}

A purpose of the description here is to re-steer/remind the panelist/reader again by tersely describing what your thesis is about (i.e. the problem and the main goal you want to achieve) in another way without sounding repetitive. 

Your methodology is your means of achieving your stated objectives.

Note that each stated objective should have a corresponding methodology of achieving it.

\graytx{\blindtext}


\ifFinished
\else

\section{Estimated Work Schedule and Budget}

The estimated work schedule can be represented as a Gantt Chart or a combination of Project Network Diagram, Work Breakdown Structure, and Critical Path.  The budget can be made into a Bill of Materials, financial plan, or if your \documentType \ is funded and part of larger project, the cost and date for reaching each milestone and/or deliverable for your part of the project.

For ECE undergraduate theses, the individual Gantt Chart and Bill of Materials will be included in this section and be removed in the final document.

\graytx{\blindtext}

\ifPhD
\section{Publication Plan}
\graytx{\blindtext}
\fi

\fi


\section{Overview of the \documentType}

Provide here a brief summary and what the reader should expect from each succeeding chapter.  Show how each chapter is connected with each other.


\stopcontents[chapters]
\cleardoublepage

%%%%%%%%%%%%%%%%%%%%%%%%%%%%%%%%%%%%%%%%%%%%%%%%
\chapter{Literature Review} 
\label{ch:litrev} 
\startcontents[chapters]
\begin{SingleSpace}	
	\Mprintcontents 
\end{SingleSpace}
It is to be noted that each subsection in this chapter should discuss in narrative form each table that is presented.  

\section{Existing Work}

Cite and summarize here relevant and significant literature (dissertations, theses, journals, patents, notable conference papers) through a table and descriptions to prove that no one has done your work yet and/or that your work is not a duplication of existing ones. Your focus here is what has \emph{been done}.

\graytx{\Blindtext}

\section{Lacking in the Approaches}

You can summarize the weaknesses of existing approaches by a tabular comparison of the literature. Your focus here is what has \emph{not been done}, i.e. what features were missed, what solutions were not considered, what the demerits are, etc.  Through these items, you then can introduce the necessity for doing your proposed solution.  

It is to be noted that degree of novelty for undergraduate thesis is lower than those for graduate school. If a PhD dissertation/thesis has a high degree of novelty and that for an undergraduate is low, then a master's thesis is somewhere between the two.

Briefly include here the following in order to remind the reader why you are highlighting the weaknesses of the solutions of existing literature. 

\begin{itemize}
	\item mentioning of the problem
	\item showing how your solution is better (can be better (for proposals))
\end{itemize}


\graytx{\Blindtext}

\section{Summary}

Provide the gist of this chapter such that it reflects the contents and the message.





\stopcontents[chapters]
\cleardoublepage

%%%%%%%%%%%%%%%%%%%%%%%%%%%%%%%%%%%%%%%%%%%%%%%%
\chapter{Theoretical Considerations}
%\chaptermark{Theoretical Considerations} % uncomment this and put a shorter version of the chapter title for the TOC and chapter markings (i.e., header or footer)
\label{ch:theorycon}
\startcontents[chapters]
\begin{SingleSpace}	
	\Mprintcontents 
\end{SingleSpace}
Before starting the first section, provide an overview of the purpose of this chapter and its contents, and how they are relevant to your methodology.  Discuss in this chapter the relevant theories and concepts that should support your proposed solutions.

This chapter is for providing the context to your panelist/reader.  It is actually an expanded form of the Background of the Study that you have put in Chapter~\ref{ch:intro}.

\graytx{\Blindtext}

\begin{figure}[!htbp]
	\centering
		\includegraphics[width=0.5\textwidth]{example_gray_box}
	\caption{A quadrilateral image example.}
	\label{fig:exampletc}
\end{figure}

\section{Summary}

Provide the gist of this chapter such that it reflects the contents and the message.
\stopcontents[chapters]
\cleardoublepage

%%%%%%%%%%%%%%%%%%%%%%%%%%%%%%%%%%%%%%%%%%%%%%%%
\chapter{Design Considerations} 
\label{ch:designcon} 
\startcontents[chapters]
\begin{SingleSpace}	
	\Mprintcontents 
\end{SingleSpace}
Before starting the first section, provide an overview of the purpose of this chapter and its contents, and how they are relevant to your methodology. 

Your primary goal in the Design Considerations chapter is to describe to your panelist/readers the key topics that fall further under Theoretical Considerations, but should be placed here instead since they are geared towards your Methodology. These key topics are those that you have directly adopted in making your solution/methodology.  You can think of the connection of the Design Considerations chapter to the Theoretical Considerations chapter in this way: if your Theoretical Considerations chapter serves as the main foundation of a building, then the Design Considerations chapter functions as the columns. 

The  Design Considerations chapter is an avenue for explaining why you considered the topics here for your proposed methodology. This chapter is different from your methodology, because topics you discuss here are already accepted as part of the body of knowledge, and may have not been developed by you.  


\graytx{\Blindtext}


\section{Summary}

Provide the gist of this chapter such that it reflects the contents and message.
\stopcontents[chapters]
\cleardoublepage

%%%%%%%%%%%%%%%%%%%%%%%%%%%%%%%%%%%%%%%%%%%%%%%%
\chapter{Methodology} 
\label{ch:method} 
\startcontents[chapters]
\begin{SingleSpace}	
	\Mprintcontents 
\end{SingleSpace}
Put an overview of the contents of chapter. Mention here your methodology flow through a figure and provide an overview of it and how your methodology achieves your objectives.  How your methodology achieves each of your specific objective is what your panelists/examiners will be looking for.  Specify how your methodology achieves your general objective, and specific objectives.  A point-by-point comparison how your methodology achieves each of your specific objective is expected in the final \documentType.

Also make sure that you refer clearly to the chapters on the Literature Review, Theoretical Considerations, and Design Considerations showing how your methodology ties with those that you have discussed in those chapters.

Make an overview of the contents of chapter. Put here your methodology flow through a figure and provide an overview of it.  
\section{Implementation}
\label{sec:implement}

Summarize the process used to create/set-up the work with an explanation of such process, instruments, and materials that you used if any. If the description is lengthy, use condensed bullet points. 

\noindent \textit{Rule of thumb}: Implementation is how you made your  work; (keywords: implemented, created, made, soldered, programmed, etc.).

If you wrote a program or made a simulation, you must state how the program or simulation functions in this section.	An algorithm or a pseudocode as shown in Table~\ref{tab:calcxn} is a good example.


\graytx{\Blindtext}



\section{Evaluation}
\label{sec:evaluate}

Describe the procedures for evaluating the correct behavior and outcome of your  work, including what information you need to gather and how you will obtain or measure it.  

\textit{Rule of thumb}: Evaluation is how you tested your  work; (keywords: measured, tested, compared, simulated, etc.).

\graytx{\Blindtext}



\section{Summary}

Provide the gist of this chapter such that it reflects the contents and the message.

\stopcontents[chapters]
\cleardoublepage

%%%%%%%%%%%%%%%%%%%%%%%%%%%%%%%%%%%%%%%%%%%%%%%%
\ifResultDiscuss 
	\chapter{Results and Discussions} 
	\label{ch:result_discuss} 
	\startcontents[chapters]
	\begin{SingleSpace}	
		\Mprintcontents 
	\end{SingleSpace}
	
Show in this chapter proofs why your proposed solution works.  However, presenting results ("It worked") without an appropriate explanation does not show thorough understanding.  Aside from the data and results that you have obtained, and their explanation, the discussion includes why components of your proposed solution work did or did not work in accordance to what you described in the evaluation process, and how the proposed solution performed and faired. Interpret the results and the reasons why they were obtained.  If your results are incorrect, apparent discrepancies from theory should be pointed out and explained. In essence, what do the results mean.  Citing existing publication can help you compare your results and your explanations. 

The next items below is not related to the description of this results and discussions chapter, but serves as an opener for the \LaTeX portion of this template.

Here is an example of a citation for ISO~80000-2 standard~\cite{ISO800002}. Another one is~\cite{Einstein} and~\cite{croft-78}. 

In using this template, the user is expected to have a working knowledge of \LaTeX. A good introduction is in~\cite{Oetiker2014}.  Its latest version can be accessed at \url{http://www.ctan.org/tex-archive/info/lshort}. See the Appendix of \verb|document_guide.pdf |for examples.



\graytx{\Blindtext}

\section{Summary}

Provide the gist of this chapter such that it reflects the contents and the message.
	\stopcontents[chapters]
	\cleardoublepage
\fi

%%%%%%%%%%%%%%%%%%%%%%%%%%%%%%%%%%%%%%%%%%%%%%%%
\ifConc
	\chapter{Conclusions, Recommendations, and Future Directives} 
	\label{ch:conc} 
	\startcontents[chapters]
	\begin{SingleSpace}	
		\Mprintcontents 
	\end{SingleSpace}
	\section{Concluding Remarks}

In this \documentType, \ldots

Put here the main points that should be known and learned about the  work topic. Summarize or give the gist of the essential principles and inferences drawn from your results.

\section{Contributions}

The interrelated \index{contributions} contributions and supplements that have been developed by the author(s) in this \documentType \ are listed as follows.  Only those that are unique to the authors' work are included.

\begin{itemize}
  \item the ; 
	
	\item the ; 
  
  \item the ; 
	
\end{itemize}


\section{Recommendations}

\graytx{\Blindtext}

\section{Future Prospects}

There are several prospect related in this research that may be extended for further studies. \ldots So the suggested topics are listed in the following.

\begin{enumerate}
	\item  the \ldots.
	
	\item  the \ldots.
		
	\item  the \ldots.
\end{enumerate}

Note that for ECE undergraduate theses, as per the directions of the thesis adviser, Recommendations and Future Directives will be removed for the hardbound copy but will be retained for database storage.


	\stopcontents[chapters]
	\cleardoublepage
\fi

%%%%%%%%%%%%%%%%%%%%%%%%%%%%%%%%%%%%%%%%%%%%%%%
\renewcommand{\UrlFont}{\normalfont}
%\bibliographystyle{IEEEtr} % for IEEE referencing format
\bibliographystyle{apalike} % for APA referencing format
\begin{SingleSpace}
  {\small \bibliography{references}}
	\vfill
	\LaTeX-comment this and the following texts after you have implemented them. See the following references for helpful guides for the bibliography and script editing in general.  Note that the links might be unavailable, but the names can be searched in the Web.
		
	\begin{enumerate}
		\item IEEE Citation Reference: \url{www.ieee.org/documents/ieeecitationref.pdf}
		
		\item IEEE Editorial Style manual: \url{www.ieee.org/documents/style_manual.pdf} 
		
		\item IEEE Abbreviations for Transactions, Journals, Letters, and Magazines: \url{www.ieee.org/documents/trans_journal_names.pdf}
	\end{enumerate}
	
\noindent Also in your BibTeX file, enclose letters or words that should all be in uppercase in curly brackets. Example: {IBM}, {P}hilippines, e{X}tensible {M}arkup {L}anguage.

\end{SingleSpace}
\vfill
\begin{flushright}
Produced: \usdate\today, \currenttime \\
\end{flushright}
\cleardoublepage 

%%%%%%%%%%%%%%%%%%%%%%%%%%%%%%%%%%%%%%%%%%%%%%%%
\SingleSpacing
\appendix
\renewcommand{\thechapter}{\Alph{chapter}}
\renewcommand{\thesection}{\thechapter\arabic{section}}
\appto\appendix{\renewcommand\thechapter{\AlphAlph{\value{chapter}}}} % for increasing appendix chapters beyond Z, i.e. AA, AB, etc.

%%%%%%%%%%%%%%%%%%%%%%%%%%%%%%%%%%%%%%%%%%%%%%%%
\chapter{Student Research Ethics Clearance}
\includepdf[pages={1},%
offset=3.5mm -10mm,%
scale=0.75,%
pagecommand={},]
{./figure/STUDENT_RESEARCH_ETHICS_CLEARANCE.pdf}
\cleardoublepage

%%%%%%%%%%%%%%%%%%%%%%%%%%%%%%%%%%%%%%%%%%%%%%%%
\chapter{Answers to Questions to this \documentType}
%\startcontents[chapters]
%\Mprintcontents 



\refstepcounter{section}\section*{\thesection\quad  How important is the problem to practice?}

A possible answer to this question is the summary of your Significance of the Study, and that portion of the Problem Statement where you describe the ideal scenario for your intended audience. 

\graytx{\blindtext}
	
	
	
	
\refstepcounter{section}\section*{\thesection\quad  How will you know if the solution/s that you will achieve would be better than existing ones?}	

\graytx{\blindtext}


\refstepcounter{subsection}\subsection*{\thesubsection\quad How will you measure the improvement/s?}	

\graytx{\blindtext}

	
\refstepcounter{subsubsection}\subsubsection*{\thesubsubsection\quad  What is/are your basis/bases for the improvement/s?}

\graytx{\blindtext}
	
		
\refstepcounter{subsubsection}\subsubsection*{\thesubsubsection\quad  Why did you choose that/those basis/bases?}

\graytx{\blindtext}

				
\refstepcounter{subsubsection}\subsubsection*{\thesubsubsection\quad  How significant are your measure/s of the improvement/s?}

\graytx{\blindtext}






	
\refstepcounter{section}\section*{\thesection\quad What is the difference of the solution/s from existing ones?}
	
\graytx{\blindtext}

\refstepcounter{subsection}\subsection*{\thesubsection\quad How is it different from previous and existing ones?}

\graytx{\blindtext}
	
	
	
	
	
	
\refstepcounter{section}\section*{\thesection\quad What are the assumptions made (that are behind for your proposed solution to work)?}
	
\graytx{\blindtext}
		
	
\refstepcounter{subsection}\subsection*{\thesubsection\quad Will your proposed solution/s be sensitive to these assumptions?}
	
\graytx{\blindtext}

  
\refstepcounter{subsection}\subsection*{\thesubsection\quad Can your proposed solution/s be applied to more general cases when some of the assumptions are eliminated? If so, how?}

\graytx{\blindtext}






\refstepcounter{section}\section*{\thesection\quad What is the necessity of your approach / proposed solution/s?}

\graytx{\blindtext}
	
	
\refstepcounter{subsection}\subsection*{\thesubsection\quad What will be the limits of applicability of your proposed~solution/s?}

\graytx{\blindtext}
				
						
\refstepcounter{subsection}\subsection*{\thesubsection\quad What will be the message of the proposed solution to technical people?  How about to non-technical managers and business men?}
			
\graytx{\blindtext}





\refstepcounter{section}\section*{\thesection\quad How will you know if your proposed solution/s is/are correct?}

\graytx{\blindtext} 
			
			
\refstepcounter{subsection}\subsection*{\thesubsection\quad Will your results warrant the level of mathematics used (i.e., will the end justify the means)?}
	    
\graytx{\blindtext}
			





\refstepcounter{section}\section*{\thesection\quad Is/are there an/\_ alternative way/s to get to the same solution/s?}

\graytx{\blindtext}
	
	
\refstepcounter{subsection}\subsection*{\thesubsection\quad Can you come up with illustrating examples, or even better, counter examples to your proposed solution/s?}

\graytx{\blindtext}
	
	
\refstepcounter{subsection}\subsection*{\thesubsection\quad Is there an approximation that can arrive at the essentially the same proposed solution/s more easily?}
	
\graytx{\blindtext}
			
	
	
	
	
\refstepcounter{section}\section*{\thesection\quad If you were the examiner of your \documentType, how would you present the \documentType \ in another way?  Give your remarks, especially for your methodology and the results and discussions.}

% \MakeTextLowercase{\documentType} currently fails inside section{}
	
\graytx{\blindtext}
	
	
\refstepcounter{subsection}\subsection*{\thesubsection\quad What are the weaknesses of your \documentType, specifically  your methodology and the results and discussions?}

\graytx{\blindtext}

%\stopcontents[chapters]
\cleardoublepage

%%%%%%%%%%%%%%%%%%%%%%%%%%%%%%%%%%%%%%%%%%%%%%%%
\chapter{Revisions to the Proposal} 
\label{ch:revisions_to_the_proposal}
%\startcontents[chapters]
%\Mprintcontents  
Make a table with the following columns for showing the summary of revisions to the proposal based on the comments of the panel of examiners. 
\begin{enumerate}
	\item  Panelist name
	\item  Comment
	\item  Summary of how the comment was addressed
	\item  Locations in the document where the changes have been reflected
\end{enumerate}


\begin{center}
{\scriptsize
\begin{tabularx}{\textwidth}{p{0.1\textwidth}|p{0.2\textwidth}|p{0.5\textwidth}|p{0.1\textwidth}}
\caption{Summary of Revisions to the Proposal} \label{tab:rev_proposal} \\
\hline 
\hline 
\textbf{Panelist name} & 
\textbf{Comment} & 
\textbf{Summary of how the comment was addressed} &
\textbf{Locations} \\ 
\hline 
\endfirsthead
\multicolumn{4}{c}%
{\textit{Continued from previous page}} \\
\hline
\hline 
\textbf{Panelist name} & 
\textbf{Comment} & 
\textbf{Summary of how the comment was addressed} &
\textbf{Locations} \\  
\hline 
\endhead
\hline 
\multicolumn{4}{r}{\textit{Continued on next page}} \\ 
\endfoot
\hline 
\endlastfoot

\documentAdviserTitle\ \documentAdviser &
\graytx{\blindtext} &
\graytx{\blindtext \blinddescription} &
Sec.~\ref{sec:implement} on p.~\pageref{sec:implement}, Sec.~\ref{sec:evaluate} on p.~\pageref{sec:evaluate}, Fig.~\ref{fig:exampletc} on p.~\pageref{fig:exampletc}\\
\hline \\

\examinerChairTitle\ \examinerChair & 
\graytx{\blindtext} &
\graytx{\blindtext \blinddescription} &
Sec.~\ref{sec:implement} on p.~\pageref{sec:implement}, Sec.~\ref{sec:evaluate} on p.~\pageref{sec:evaluate}, Fig.~\ref{fig:exampletc} on p.~\pageref{fig:exampletc}\\
\hline \\

\examinerATitle\ \examinerA & 
\graytx{\blindtext} &
\graytx{\blindtext \blinditemize} &
Sec.~\ref{sec:implement} on p.~\pageref{sec:implement}, Sec.~\ref{sec:evaluate} on p.~\pageref{sec:evaluate}, Fig.~\ref{fig:exampletc} on p.~\pageref{fig:exampletc}\\
\hline \\

\examinerBTitle\ \examinerB & 
\graytx{\blindtext} &
\graytx{\blindtext \blindenumerate} &
Sec.~\ref{sec:implement} on p.~\pageref{sec:implement}, Sec.~\ref{sec:evaluate} on p.~\pageref{sec:evaluate}, Fig.~\ref{fig:exampletc} on p.~\pageref{fig:exampletc}\\
\hline \\

\examinerCTitle\ \examinerC & 
\graytx{\blindtext} &
\graytx{\blindtext \blindmathtrue} &
Sec.~\ref{sec:implement} on p.~\pageref{sec:implement}, Sec.~\ref{sec:evaluate} on p.~\pageref{sec:evaluate}, Fig.~\ref{fig:exampletc} on p.~\pageref{fig:exampletc}\\
\hline \\

\end{tabularx}
}
\end{center}
%\stopcontents[chapters]
\cleardoublepage

%%%%%%%%%%%%%%%%%%%%%%%%%%%%%%%%%%%%%%%%%%%%%%%%
\chapter{Revisions to the Final} 
\label{ch:revisions_to_the_final}
%\startcontents[chapters]
%\Mprintcontents  
\include{revisions_to_the_final}
%\stopcontents[chapters]
\cleardoublepage

%%%%%%%%%%%%%%%%%%%%%%%%%%%%%%%%%%%%%%%%%%%%%%%%
\chapter{Usage Examples} 
\label{ch:usage_examples}
%\startcontents[chapters]
%\Mprintcontents  
\include{usage_examples}
%\stopcontents[chapters]
\cleardoublepage

%%%%%%%%%%%%%%%%%%%%%%%%%%%%%%%%%%%%%%%%%%%%%%%%
\ifPubList
	\chapter{Publication List and Award}

\flushleft{\Large \bfseries Journal (example only)\\}

\begin{enumerate}
\item \href{http://10.1016/j.jorganchem.2006.03.012}{{\"O}.~Aks{\i}n, H.~T{\"u}rkmen, \textbf{L.~Artok}, B.~{\c{C}}etinkaya, C.~Ni, 
  O.~B{\"u}y{\"u}kg{\"u}ng{\"o}r, and E.~{\"O}zkal, ``Effect of immobilization
  on catalytic characteristics of saturated pd-n-heterocyclic carbenes in
  mizoroki-heck reactions,'' {\em Journal of Organometallic Chemistry},
  vol.~691, no.~13, pp.~3027--3036, 2006.}

\item \ldots

\end{enumerate}
\vspace{2ex}


\flushleft{\Large \bfseries Conference\\}

\begin{enumerate}

\item \ldots

\item \ldots

\end{enumerate}
\vspace{2ex}



\flushleft{\Large \bfseries Others}\\
\begin{enumerate}

\item \ldots

\item \ldots

\end{enumerate}
\vspace{2ex}



\flushleft{\Large \bfseries Award}\\

\begin{enumerate}
\item \ldots

\item \ldots
\end{enumerate}
\fi
\cleardoublepage

%%%%%%%%%%%%%%%%%%%%%%%%%%%%%%%%%%%%%%%%%%%%%%%%
\ifVita
	\include{vita}
\fi
\cleardoublepage

%%%%%%%%%%%%%%%%%%%%%%%%%%%%%%%%%%%%%%%%%%%%%%%%
\ifIndex
	\printindex
\fi
\cleardoublepage

\end{document}